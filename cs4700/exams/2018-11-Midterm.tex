\documentclass{USUexam}

\setheader{CS4700}{Programming Languages}{Midterm}
\date{November 1, 2018}

\begin{document}
\shortanswer{5}{What does it mean for two language features to be orthogonal?  Give an example.}
\shortanswer{10}{Name a language criterion, why is it a useful metric for langauge selection?}
\shortanswer{6}{List the three operators allowed in a regular expression.}

\programming{20}{Write a context free grammar for the language of strings with more a's than b's.  The a's and b's can be in any order.  e.g(a,aba,aab,babbaaa are all elements of the language, $\lambda$, b, ab, abb are not).}

\shortanswer{4}{Describe two different definitions of type equivalence.}
\shortanswer{5}{What is a type error?}
\shortanswer{5}{What is meant by a `strongly typed' language?}

\shortanswer{5}{What does binding mean? List three times it can occur.}
\shortanswer{10}{What is scope?}
\shortanswer{10}{What is aliasing?}

\programming{20}{Write a set of prolog rules for determining if two lists intersect (contain at least one element in common).}

\end{document}
